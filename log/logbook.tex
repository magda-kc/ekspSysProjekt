\documentclass{article}
\usepackage{amsthm}
\usepackage{bm}
\usepackage{amsmath}
\usepackage{graphicx}
\usepackage{amsfonts}
\usepackage{newpxtext,newpxmath}
\usepackage{scrextend}
\usepackage{seqsplit}
\usepackage[autostyle]{csquotes}  
\usepackage{xspace}
\usepackage[margin=1.25in]{geometry}
\usepackage{wasysym}
\usepackage{pdfpages}
\usepackage[toc,page]{appendix}
\usepackage{relsize}
\usepackage{tikz}
\usepackage{pgfplots} 
% Indent all description environments slightly
\usepackage{enumitem}
\setlist[description]{leftmargin=\parindent,labelindent=\parindent}

\usepackage{todonotes}

\newtheorem{definition}{Definition}
\newtheorem{theorem}{Theorem}

\newcommand{\horrule}[1]{\rule{\linewidth}{#1}} % Create horizontal rule command with 1 argument of height\newcommand{\vsp}[1]{\vspace{#1} \\}
\newcommand{\hsp}[1]{\-\hspace{#1}}

% Use dash instead of bullets for itemize
\renewcommand\labelitemi{--}

\title{Logbook}
\author{Jessie Chatham Spencer \and Magdalena Kalin}


\date{\today}
\begin{document}
\maketitle
\subsection*{16-03-2020}
The goal of todays session is to come up with some ideas for what process we want to improve and get a general idea of what our solution will be. 

We brainstormed which parts of our lives could be made easier hanks to an IT-solution. Each one of us then wrote some ideas down for what we would like. This prove to be harder than expected, as at first it was not easy to identify even a single problem, however as some time passed and we took a closer look at our lives, and were able to come up with some ideas. But as we were studying our day very closely, maybe even too closely to notice the bigger picture, it became hard to think of more ideas. We then also tried looking at other people in our lives and see if we noticed something they might want improved to see if this is something we also would find useful, but were not able to spot in our lives as we lacked a third person perspective.

We then compared these ideas and found some that both of us would be fairly interested in even if it would benefit one person more than the other. We then talked about these ideas and thought of general ideas of how we would want them to look.

\subsection*{18-03-2020}
We have conducted some observations to let us figure out who exactly is our primary target group, as well as when the app would be necessary and what kind of activities would need it.

At our local gym, we looked at how those who record their workouts do it. We observed what kind of information they write down, how often they do it and for which part of their workouts they do it.

Here, we identified that our target groups are people who already have a workout plan, know what they're doing and would like an easier way to write it down than carrying a notebook around. However, another group of target users was personal coaches. They take charge of the workout plan and what kind of exercises their clients do. We also noticed that this solution could be usefu for newbies, who want to start trackign their progress, but don't know how. This app coudl help them start so that it is not as overwhelming as doing a bunch of research or being unsure of what to note down.

\subsection*{27-03-2020}
During this session, we aimed to further specify our idea for the solution, as well as plan the workload and how we will tackle it.

We started working on our affinity model, as this was the easiest way for us to gather and narrow down ideas. Each one of us sat down and wrote down a bunch of ideas so that we could compare, group and think about how we would include these ideas in our solution. It turned out that a lot of ideas were similar, and some even not fully fitting to the project. This however meant that we gained a deeper understanding of what our solution is as we could also learn what we did not want in our app.

Today, working remotely proved to be a small challenge, as we were not able to see what the other person was doing and what ind of ideas they were coming up with. It was also harder, as there were some misunderstandings. However, all went well as we worked on resolving these misunderstandings and gaining new knowledge of how we want to develop the idea.

Once we were done with this, we also looked at how we want to split up the workload so that each one of us gets equally much out of this project. We created our backlog, so that we can get an overview of how much work needs to be done and so that we can start working on the different tasks.

\subsection*{06-03-2020}
Today, we wanted to work on including the things we worked on in our report, as well as work on our status report. Here, working remotely became a bit of an obstacle once again, as we wanted to share what we had written down on paper or show something to the other group member, and it was harder than expected, as we had to take pictures or screenshots of what we wanted to show. This also meant that if we wanted to demonstrate something by drawing it or writing it down, we had to do that first and then share it with the other person.

\subsection*{11-04-2020}
We want to create our initial specification today and get an overview of how far we are with the status report.

We made an attempt writing the initial specification, but we had a hard time figuring out what an initial specification actually means. We also added ideas to our affinity diagram and discussed ideas for how the prototype could function. We tried to visualize what the what kind of functionalities we want the prototype to have, as well as what we want it to look like, so that it is easiest for the users.

\subsection*{29-04-2020}
During this meeting, we conducted an interview (since we are now the expert users), to determine what we want to get out of this app and hear what the users want to be able to do specifically.

Here is a transcript of the interview:

M: How often do you workout?\\
J: Four to six times a week.
\\ \\
M: What kind of information do you write down?\\
How many reps and how much weight of each exercise.
J: I also ŕecord my personal records.
\\ \\
M: What do you use to record your performance?\\
J: While exercising I will write on a notepad. Afterwards I will write the information on the notepad into a spreadsheet.
\\ \\
M: How do you use the recorded data?\\
J: Basically to make sure I am improving and see if a workout plan is working. It also helps to stay motivated when I can look back and see that I have actually been following the plan and therefore to not break my streak I should continue.
\\ \\
M: What do you want to get out of this app?\\
J: I want faster and mor structured input while working out and a better overview of how I have progressed.
\\ \\
M: How often do you expect to use the app?\\
J: Atleast everytime I work out and then also when planning workouts or just flexing my progress. So probably daily.
\\ \\
M: What kind of exercise do you expect to be inputting?\\
J: Strength training.
\\ \\
M: What is most important when you open the app/start exercising?\\
J: When opening the application I want to see my latest results over the past week or month. While exercising I just want to have the input ready for me to quickly record my sets.
\\ \\
M: When creating a workout plan how much should the user be able to input themselves and how much should be assumed by the app?\\
J: Well ideally it would be nice if there was sane defaults with the option to customize. For example by default if you could just select which days per week, but if you wanted you could customize the days over a monther or two weeks. Usually I make my plans per week, where each day is different or there are pairs of days that have the same workout during the week. Like two push days and two pulls days.
Ofcourse for each day I want to write down exercises with a target number of sets and repitions.
\\ \\
M: Should the app suggest exercises based on the workout plan and target muscle groups (if this info is input)?\\
J: I think that would be useful for people starting out, but most of the time I think I would like to just input my plan and move on. A catalog of exercises with instructions would be nice.\\

To tackle our ``working remotely'' problem, we had this interview over Discord. This allowed for us to talk easily and effectively, as well as discuss different aspects of the app, which we may not have done if this interview was done through text.

\subsection*{04-05-2020}
We took turns evaluating the prototype. Since we are already far apart, we both found a time during our day to evaluate the app on our own before we scheduled to meet up over Discord. We then sent our lists of feedback to each other so that we coudl read them and then discuss. This worked very well, as both of us had some important points.''
A combined list of feedback can be seen below:
\begin{itemize}
\item Creating workout plan is very easy
\item Easy to edit the exercises on the workout plan
\item Easy overview of which days of the week are for which exercises
\item Add overview of data
\item Add feedback for pressing on log button when recording
\item Add delete button for exercises in workout plan
\item Fix weight input field under record (weight input not working)
\item Overall, great design
\end{itemize}

\subsection*{05-05-2020}
After quickly fixing the minor details, we shared this app with our friends and family, some of which are very much into fitness, others not so much. This did not happen remotely, as each one of us asked our family and our neighbors to help out. We preformed a KLM evaluation on them and the app to see how it would work on real users - not just us, as we already knew which button did what and where to press for the different functionalities. We wanted to get a real overwiew of the app. This was a very positive experience for us, as it showed that the majority of the users had o probkem navigating our app.

\subsection*{13-05-2020}
Our last evaluation meeting consisted of us mainly discussing what could be done in the future. Again, this happened over Discord. We mainly discussed what could be added in the future, as we could no longer see things that needed editing (not adding) in our prototype. A list of this feedback can be seen below:
\begin{itemize}
\item Consider adding a week overview, to even more easily see when to do what (optimal if there are many exercises in workout plan)
\item Edit weight units in settings
\item Add option for creating an account
\item Store data in a better way
\item Add workout plan catalog so users don't have to create their own if they donøt want to
\item Add exercise catalog, so that it makes sense to have the autocomplete function for exercises
\end{itemize}
\end{document}
