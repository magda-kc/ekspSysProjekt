\documentclass{article}
\usepackage{amsthm}
\usepackage{bm}
\usepackage{amsmath}
\usepackage{graphicx}
\usepackage{amsfonts}
\usepackage{newpxtext,newpxmath}
\usepackage{scrextend}
\usepackage{seqsplit}
\usepackage[autostyle]{csquotes}  
\usepackage{xspace}
\usepackage[margin=1.25in]{geometry}
\usepackage{wasysym}
\usepackage{pdfpages}
\usepackage[toc,page]{appendix}
\usepackage{relsize}
\usepackage{tikz}
\usepackage{pgfplots} 
% Indent all description environments slightly
\usepackage{enumitem}
\setlist[description]{leftmargin=\parindent,labelindent=\parindent}

\newtheorem{definition}{Definition}
\newtheorem{theorem}{Theorem}

\newcommand{\horrule}[1]{\rule{\linewidth}{#1}} % Create horizontal rule command with 1 argument of height
\newcommand{\ID}{\texttt{ID}\xspace}
\newcommand{\vsp}[1]{\vspace{#1} \\}
\newcommand{\hsp}[1]{\-\hspace{#1}}

% Use dash instead of bullets for itemize
\renewcommand\labelitemi{--}

\title{Logbook}
\author{Jessie Chatham Spencer \and Magdalena Kalin}


\date{\today}
\begin{document}
\maketitle
\subsection*{16-03-2020}
The goal of todays session is to come up with some ideas for what process we want to improve and get a general idea of what our solution will be. 

We brainstormed which parts of our lives could be made easier hanks to an IT-solution. Each one of us then wrote some ideas down for what we would like. This prove to be harder than expected, as at first it was not easy to identify even a single problem, however as some time passed and we took a closer look at our lives, and were able to come up with some ideas. But as we were studying our day very closely, maybe even too closely to notice the bigger picture, it became hard to think of more ideas. We then also tried looking at other people in our lives and see if we noticed something they might want improved to see if this is something we also would find useful, but were not able to spot in our lives as we lacked a third person perspective.

We then compared these ideas and found some that both of us would be fairly interested in even if it would benefit one person more than the other. We then talked about these ideas and thought of general ideas of how we would want them to look.

\subsection*{27-03-2020}
During this session, we aimed to further specify our idea for the solution, as well as plan the workload and how we will tackle it.

We started working on our affinity model, as this was the easiest way for us to gather and narrow down ideas. Each one of us sat down and wrote down a bunch of ideas so that we could compare, group and think about how we would include these ideas in our solution. It turned out that a lot of ideas were similar, and some even not fully fitting to the project. This however meant that we gained a deeper understanding of what our solution is as we could also learn what we did not want in our app.

Today, working remotely proved to be a small challenge, as we were not able to see what the other person was doing and what ind of ideas they were coming up with. It was also harder, as there were some misunderstandings. However, all went well as we worked on resolving these misunderstandings and gaining new knowledge of how we want to develop the idea.

Once we were done with this, we also looked at how we want to split up the workload so that each one of us gets equally much out of this project. We created our backlog, so that we can get an overview of how much work needs to be done and so that we can start working on the different tasks.

\subsection*{06-03-2020}
Today, we wanted to work on including the things we worked on in our report, as well as work on our status report. Here, working remotely became a bit of an obstacle once again, as we wanted to share what we had written down on paper or show something to the other group member, and it was harder than expected, as we had to take pictures or screenshots of what we wanted to show. This also meant that if we wanted to demonstrate something by drawing it or writing it down, we had to do that first and then share it with the other person.

\subsection*{11-04-2020}
We want to create our initial specification today and get an overview of how far we are with the status report.

We made an attempt writing the initial specification, but we had a hard time figuring out what an initial specification actually means. We also added ideas to our affinity diagram and discussed ideas for how the prototype could function. We tried to visualize what the what kind of functionalities we want the prototype to have, as well as what we want it to look like, so that it is easiest for the users.
\end{document}
